%preamble.tex

\usepackage[T1]{fontenc}
\usepackage{lmodern}
\usepackage[utf8]{inputenc}
\usepackage[T1]{polski}
\usepackage{helvet}
\usepackage{graphicx}
\usepackage{color}
\usepackage{geometry}
\usepackage{epstopdf}
\usepackage{listings}
\usepackage{textcomp}
\usepackage{physics}
\usepackage{mathtools}  % loads »amsmath«
\usepackage{amsfonts}
\usepackage{enumitem}
\usepackage{hyperref} % Required for adding links and customizing them
\usepackage[dvipsnames]{xcolor}
%\usepackage{algorithm}% http://ctan.org/pkg/algorithms
\usepackage[boxed, ruled, vlined]{algorithm2e}
\usepackage[backend=biber, style=alphabetic]{biblatex}
\usepackage{filecontents}
\usepackage{import}
\usepackage{subcaption}
\usepackage{float}
\usepackage[section]{placeins}

\SetKwComment{Comment}{$\triangleright$\ }{}
\SetCommentSty{itshape}

%\ifthenelse{\boolean{algocf@optonelanguage}\AND\equal{\algocf@languagechoosen}{polish}}{%
%\SetKwInput{KwIn}{Wejście}%
%\SetKwInput{KwOut}{Wyjście}%
%\SetKwInput{KwData}{Dane}%
%\SetKwInput{KwResult}{Rezultat}%
%\SetKw{KwTo}{Do}%
%\SetKw{KwRet}{devolver}%
%\SetKw{Return}{return}%
%\SetKwBlock{Begin}{inicio}{fin}%
%\SetKwRepeat{Repeat}{repetir}{hasta que}%
%
%\SetKwIF{If}{ElseIf}{Else}{Jeżeli}{entonces}{sin\'o, si}{en otro caso}{end if}
%\SetKwSwitch{Switch}{Case}{Other}{seleccionar}{hacer}{caso}{sin\'o}{fin caso}{fin seleccionar}
%\SetKwFor{For}{per}{fai}{fine per}%
%\SetKwFor{ForPar}{par}{hacer in paralelo}{fin para}%
%\SetKwFor{ForEach}{dla każdego}{->}{end foreach}
%\SetKwFor{ForAll}{dla wszystkich}{->}{end forall}
%\SetKwFor{While}{kiedy}{->}{end while}
%}{}%

%---------------------------------- kolory hiperlaczy i linkow
\newcommand\myshade{85}
\colorlet{mylinkcolor}{violet}
\colorlet{mycitecolor}{YellowOrange}
\definecolor{myurlcolor}{rgb}{0,0.2,0.6}

\hypersetup{
  linkcolor  = mylinkcolor!\myshade!black,
  citecolor  = mycitecolor!\myshade!black,
  urlcolor   = myurlcolor!\myshade!black,
  colorlinks = true,
  breaklinks = true,
}
%----------------------------------

\graphicspath{ {./plots/} {./diagram/} {./graphics/} }

\lstset{inputpath=code}

\geometry{hmargin={2cm, 2cm}, height=10.0in}

\renewcommand*{\lstlistlistingname}{Spis kodów źródłowych}

\lstset{ %
language=C++,                % choose the language of the code
basicstyle=\footnotesize,       % the size of the fonts that are used for the code
numbers=left,                   % where to put the line-numbers
numberstyle=\footnotesize,      % the size of the fonts that are used for the line-numbers
stepnumber=1,                   % the step between two line-numbers. If it is 1 each line will be numbered
numbersep=5pt,                  % how far the line-numbers are from the code
backgroundcolor=\color{white},  % choose the background color. You must add \usepackage{color}
showspaces=false,               % show spaces adding particular underscores
showstringspaces=false,         % underline spaces within strings
showtabs=false,                 % show tabs within strings adding particular underscores
frame=single,           % adds a frame around the code
tabsize=2,          % sets default tabsize to 2 spaces
captionpos=b,           % sets the caption-position to bottom
breaklines=true,        % sets automatic line breaking
breakatwhitespace=false,    % sets if automatic breaks should only happen at whitespace
escapeinside={\%*}{*)}          % if you want to add a comment within your code
}

\begin{filecontents}{bibliography.bib}
@article{muller03,
       author = "Matthias Müller and David Charypar and Markus Gross",
       title = "Particle-Based Fluid Simulation for Interactive Applications",
       year = "2003" }
@article{kelager06,
       author = "Micky Kelager",
       title = "Lagrangian Fluid Dynamics Using Smoothed Particle Hydrodynamics",
       year = "2006" }
@article{goswami10,
       author = "Prashant Goswami and Philipp Schlegel and Barbara Solenthaler and Renato Pajarola",
       title = "Interactive SPH Simulation and Rendering on the GPU",
       year = "2010" }
@article{ihmsen13,
       author = "Markus Ihmsen",
       title = "Particle-based Simulation of Large Bodies of Water with Bubbles, Spray and Foam",
       year = "2013" }
@article{teschner14,
       author = "Markus Ihmsen and Jens Orthmann and Barbara Solenthaler and Andreas Kolb and Matthias Teschner",
       title = "SPH Fluids in Computer Graphics -- State of the Art Report",
       year = "2014" }
@article{teschner10,
       author = "Markus Ihmsen and Nadir Akinci and Marc Gissler and Matthias Teschner",
       title = "Boundary handling and adaptive time-stepping for PCISPH",
       year = "2010" }
@article{hadwiger05,
       author = "Markus Hadwiger and Christian Sigg and Henning Scharsach and Katja Bühler and Markus Gross",
       title = "Real-Time Ray-Casting and Advanced Shading of Discrete Isosurfaces",
       year = "2005" }
@article{jamriska10,
       author = "Ondrej Jamriska",
       title = "Interactive Ray Tracing of Distance Fields",
       year = "2010" }
@online{wiki:1,
  author = "Wikipedia",
  title =  "Operator Stokesa --- {W}ikipedia{,} The Free Encyclopedia",
  url = "https://pl.wikipedia.org/wiki/Operator_Stokesa",
  year = "2015", 
  note = "[Online; dostęp 19 Listopada 2015]"}
@online{wiki:2,
  author = "Wikipedia",
  title =  "Verlet integration --- {W}ikipedia{,} The Free Encyclopedia",
  url = "https://en.wikipedia.org/wiki/Verlet_integration",
  year = "2015", 
  note = "[Online; dostęp 22 Września 2015]"}
@online{wiki:3,
  author = "Wikipedia",
  title =  "Z-order curve --- {W}ikipedia{,} The Free Encyclopedia",
  url = "https://en.wikipedia.org/wiki/Z-order_curve",
  year = "2015", 
  note = "[Online; dostęp 7 Kwietnia 2015]"}
@online{@haferburg,
  author = "Andreas Haferburg",
  title =  "Object Order Empty Space Skipping in OpenGL 4",
  url = "http://haferburg.github.io/",
  year = "2015", 
  note = "[Online; dostęp 30 Października 2015]"}

\end{filecontents}

\addbibresource{bibliography.bib}
