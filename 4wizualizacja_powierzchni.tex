% 4wizualizacja_powierzchni
\newpage

\section{Wizualizacja płynu}

\paragraph{}
Wizualizacja płynu opiera się na metodzie opisanej w pracy \cite{goswami10}. W tym podejściu powierzchnia płynu reprezentowana jest przez pole wokseli. Każdy woksel zawiera dystans do najbliższej powierzchni. W ten sposób reprezentuje tzw. mapę odległości (ang. distance field). <<foto DF w 2d>>
Głównymi elementami opisywanej metody wizualizacji są: znalezienie cząsteczek na powierzchni płynu, konstrukcja mapy odległości oraz przeprowadzenie raycastingu.
\par

\subsection{Znalezienie cząsteczek powierzchniowych}

\paragraph{}
Cząsteczki, które nie leżą na powierzchni płynu nie są potrzebne do jego wizualizacji i mogą zostać pominięte. Znacznie zmniejsza to koszt generacji mapy odległości za czym idzie przyśpieszenie procesu wizualizacji.
\par
Znalezienie cząsteczek leżacych na powierzchni można przeprowadzić różnymi metodami. Jedna z nich wykorzystuje gradient pola kolorowego (ang. color field) wykorzystywane przy obliczaniu napięcia powierzchniowego płynu \cite{kelager06}. Kolejna opiera się na obliczaniu odległości danej cząsteczki do punktu środka masy (ang. center of mass). Poniżej zostały opisane obie wymienione metody.

\subsubsection{Metoda odległości do punktu środka masy}

\paragraph{}
Cząsteczka $i$ jest zaliczana do cząsteczek powierzchniowych jeżeli jej dystans do środka masy $\boldsymbol{r}_{CM_{i}}$ jej sąsiedztwa jest większy niż pewna określona wartość. Środek masy można obliczyć poprzez zsumowanie pozycji wszystkich sąsiadujących cząsteczek (w układzie odniesienia $i$-tej cząsteczki) ważone przez masę cząsteczek:
\begin{equation}
\boldsymbol{r}_{CM_{i}} = {{\sum\limits_{j} {m_j \boldsymbol{r}_{j}}} \over {\sum\limits_{j} {m_j}}}
\label{eqn:center_mass}
\end{equation}
Okazuje się jednak, że powyższy warunek jest niewystarczający w miejscach gdzie występuje niewiele cząsteczek. Dodatkowym warunkiem jest automatyczne zaliczenie cząsteczki do powierzchniowych gdy liczba cząsteczek w jej sąsiedztwie jest poniżej pewnego progu.
\par
Jeżeli cząsteczka spełnia powyższe warunki, wpisywana jest do odpowiedniej tablicy. W przeciwnym wypadku jest po prostu pomijana.
\par
\subsubsection{Metoda wykorzystująca gradient pola kolorowego}

\paragraph{}
Pole kolorowe pochodzi od funkcji \eqref{eqn:color_field}, która definiuje kształt symulowanego płynu. Korzystając z gradientu tego pola można zdefiniować normalne powierzchni płynu. Gradient ten rośnie najbardziej w kierunku `na zewnątrz' płynu. Cząsteczki powierzchniowe to te, w których miejscu długość normalnych do powierzchni przekracza pewien próg.
\begin{equation}
|\boldsymbol{n}| > t_{normal}
\end{equation}
\par
Gradient pola kolorowego obliczany jest już przy modelowaniu siły napięcia powierzchniowego \eqref{subsubsec:surface_tension}. Z tego powodu klasyfikację cząstek można przeprowadzić już na tamtym etapie, bez dodatkowych obliczeń.
\par

\subsection{Konstrukcja mapy odległości}

\paragraph{}
Mapa odległości zbudowana jest z jednorodnej trójwymiarowej siatki wokseli. Gęstość tej siatki ma istatne znaczenie dla wydajności jak i jakości odwzorowania płynu. Siatka o większej ilości wokseli pozwala na zachowanie większej ilości szczegółów ale wymaga więcej czasu na zbudowanie - czas ten rośnie w tempie $O(n^3)$. Wpływ na szybkość konstrukcji mapy ma również ilość cząsteczek powierzchniowych.
\par
Podejście proponowane przez \cite{goswami10} wygląda następująco: siatka wokseli inicjowana jest wartością maksymalną $r_{max}$. Dana cząsteczka $i$ otaczana jest sześcianem o boku długości $r$. Dla kadego woksela znajdującego się wewnątrz tego sześcianu obliczany jest dystans $d$ do cząsteczki $i$. Wartość woksela określana jest według następującego wzoru:
\begin{equation}
d_v = min(d, d_v^{old})
\label{eqn:distance_field}
\end{equation}
Jeśli $d$ jest mniejsze niż $r_{min}$, to wokselowi zostaje przypisana wartość $r_{min}$.
\par


\subsection{Renderowanie - GPU raycasting}

\paragraph{}
W ostatnim kroku wizualizacji płynu dane zawarte w mapie odległości (przechowywanej w pamięci karty graficznej jako tekstura 3D) poddawane są procesowi raycastingu \cite{hadwiger05}. Raycasting polega na przeprowadzeniu promienia przez renderowany obszar dla każdego fragmentu (piksela) obrazu; Dla każdego piksela wykonywany jest mały program - fragment shader - który inicjuje promień zberający krok po kroku próbki z danej objętości. Początek i koniec promienia muszą być znane przed uruchomieniem shadera. Dane te uzyskuje się poprzez uprzednie stworzenie bryły ograniczającej objętość renderowanego płynu. W najprostrzym przypadku bryłą tą jest prostopadłościan. Bryłę tą poddaje się procesowi rasteryzacji (w dwóch przejściach [ang. rendering pass] renderując raz przednią a raz tylną ścianę) do dwóch tekstur. Obliczając różnicę kolorów pomiędzy tymi dwoma teksturami można uzyskać kierunek promienia dla każdego piksela ograniczonego tą bryłą.
\par
<<foto bryła>>
\par
Poprzez konstrukcję bryły lepiej dopasowanej do objętości płynu można uniknąć wysyłania dużej części promieni, które nigdy nie przetną tej objętości. Istotnie zwiększa to wydajność raycastingu. Technika ta została opisana w \cite{@haferburg}.
\par
Częstotliwość próbkowania mapy odległości wzdłuż promienia ma wpływ na prędkość procesu raycastingu jak i na jakość uzyskiwanego obrazu.
\par