% 7podsumowanie
\newpage

\section{Podsumowanie, wnioski i dalszy rozwój}

\paragraph{}
W tej pracy przedstawiona została metoda symulacji płynów Smoothed Particle Hydrodynamics. Została ona wykorzystana do dyskretyzacji równania Naviera-Stokesa dla mechaniki płynów. W symulacji użyto opisu płynu metodą Lagrange'a wykorzystującego cząsteczki. Przedstawione zostały oddziaływania charakteryzujące mechanikę cieczy: ciśnienie, lepkość, napięcie powierzchniowe.
\par
Do przyśpieszenia obliczeń wykorzystano mechanizm wyszukiwania sąsiadów bazujący na jednorodnej siatce. Siatka ta dzieli obszar symulacji i jednocześnie wyznacza możliwych sąsiadów poszczególnych cząsteczek. Aby zoptymalizować dostęp do pamięci (zmniejszyć ilość błędnych odniesień do pamięci cache, ang. cache miss), uwzględniono sortowane cząsteczek według krzywej mortona (ang. Z-curve).
\par
Zaimplementowano sposób wizualizacji symulowanych cząsteczek bazujący na trójwymiarowej mapie odległości (ang. distance field) opisującej kształt powierzchni płynu. Po zbudowaniu mapy przeprowadzany jest na karcie graficznej proces ray castingu.
\par
Opisana metoda pozwala na przeprowadzenie symulacji i wizualizacji kilku tysięcy cząstek do użytku w czasie rzeczywistym (dla 4000 cząstek jedna iteracja trwa 160 ms na testowej konfiguracji). Istnieje duże pole do optymalizacji, głównie w zakresie poprawienia lokalności danych. Duży wzrost wydajności - przy odpowiedniej implementacji o rząd wielkości - przyniesie wykorzystanie GPU do przetwarzania symulacji.
\par
Zbadano jaki wpływ na stabilność symulacji mają parametry: lepkość i krok czasowy.
\par
Porównano jakościowe oraz wydajnościowe wyniki wizualizacji w zależności od gęstości próbkowania płynu na mapie odległości.
\par
