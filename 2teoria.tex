% 2Teoria
\newpage

\section{Teoria}

\paragraph{}
Metoda SPH zastosowana przy symulacji płynów służy m.in. do dyskretyzacji równania Naviera-Stokesa.
\par

\subsection{Równanie Naviera-Stokesa}
\label{subsec:navier_stokes_ss}

\paragraph{}
Płyn opisuje się przy pomocy 3 parametrów: prędkości $v$, gęstości $\rho$ i ciśnienia $p$. W celu opisu wykorzystać można również dwa podejścia charakteryzujące ośrodek: Eulera i Lagrange'a. W tym pierwszym płyn umieszcza się na siatce i jego parametry są obliczane w poszczególnych komórkach tej siatki. Za pomocą wspomnianych trzech parametrów opisuje się postępowanie płynu w czasie za pomocą dwóch równań. Pierwsze równanie opisuje zasadę zachowania masy:
\begin{equation}
\frac{\partial {\rho}}{\partial t} + \nabla \boldsymbol{\cdot} (\rho \boldsymbol{v}) = 0.
\label{eqn:mass_conservation}
\end{equation}

Drugie równanie czyli równanie Naviera-Stokesa opisuje zasadę zachowania pędu (drugą zasadę dynamiki Newtona) w nieściśliwym płynie.

{\sc Równanie Naviera--Stokesa:}
\begin{equation}
\rho \left( \frac{\partial \boldsymbol{v}}{\partial t} + \left( \boldsymbol{v} \boldsymbol{\cdot} \boldsymbol{\nabla} \right) \boldsymbol{v} \right) = -\boldsymbol{\nabla} p + \mu \nabla^{2} \boldsymbol{v} + \boldsymbol{F}
\label{eqn:navier_stokes}
\end{equation}
gdzie $\boldsymbol{F}$ oznacza tzw. siły zewnętrzne oddziałujące na płyn a $\mu$ to współczynnik lepkości dynamicznej płynu. Z tego równania wyprowadzimy wyrażenia na siły lepkości i ciśnienia występujące w symulowanym płynie.\\
Przy opisie Lagrange'a wykorzystuje się cząsteczki. Każda cząsteczka reprezentuje pewną stałą objętość płynu przypisaną do tej cząsteczki i opisaną wspomnianymi powyżej trzema parametrami. Suma wszystkich cząsteczek w całości definiuje płyn. Takie założenia znacząco ułatwiają równania opisujące płyn. Liczba cząsteczek podczas symulacji jest stała co gwarantuje spełnienie zasady zachowania masy i rówania \eqref{eqn:mass_conservation}. Dodatkowo wyrażenie z lewej strony równania \eqref{eqn:navier_stokes}: $\left( \frac{\partial \boldsymbol{v}}{\partial t} + \left( \boldsymbol{v} \boldsymbol{\cdot} \boldsymbol{\nabla} \right) \boldsymbol{v} \right)$ jest równoważne pochodnej materialnej ${D\boldsymbol{v} \over Dt}$ \cite{wiki:1}. W opisie Lagrange'a układ odniesienia powiązany jest z cząsteczką, która porusza się z płynem. Dzięki temu stwierdzeniu wyrażenie upraszcza się do pochodnej ${d\boldsymbol{v} \over dt}$.
\par

\paragraph{}
Z prawej strony równania \eqref{eqn:navier_stokes} pozostają trzy składniki. Wyrażenie $\left( \boldsymbol{\nabla} p \right)$ opisuje gradient ciśnienia, $\left( \mu \nabla^{2} \boldsymbol{u} \right)$ opisuje lepkość a $\left( \boldsymbol{F} \right)$ to pozostałe siły zewnętrzne (np. grawitację). Równanie \eqref{eqn:navier_stokes} możemy ostatecznie zapisać jako:
\begin{equation}
\boldsymbol{f}_i = -\boldsymbol{\nabla} p_i + \mu \nabla^{2} \boldsymbol{u}_i + \rho_i \boldsymbol{g}.
\label{eqn:navier_stokes_lagrangian}
\end{equation}
Zgodnie z drugą zasadą dynamiki wartość działających sił jest proporcjonalna do zmiany pędu w czasie. Zakładając stałość masy podczas ruchu, przyśpieszenie każdej z cząsteczek można znaleźć następująco:
\begin{equation}
\boldsymbol{a}_i = {d\boldsymbol{v}_i \over dt} = {\boldsymbol{f}_i \over \rho_i}.
\label{eqn:acc}
\end{equation}
Tak obliczone przyspieszenie posłuży nam do znalezienia pozycji i prędkości w chwili czasowej dla każdej cząsteczki. Metody całkowania zostaną opisane w rozdziale \eqref{subsec:integration_ss}.
\par

\subsection{Metoda SPH}
\paragraph{}
Metoda SPH została stworzona niezależnie przez R.A. Gingolda i J.J. Monaghana (1977) oraz J.B. Lucy (1977) do symulacji problemów astrofizycznych jednak jest na tyle ogólna, że z powodzeniem znalazła zastosowanie w innych typach modelowań, m.in. w symulacji płynów. Po raz pierwszy SPH do symulacji płynów wykorzystał \cite{muller03}. Od tamtego czasu ewoluowała w stronę bardziej skomplikowanych algorytmów m.in.: WCSPH, PCISPH, IISPH, FLIP.\\
SPH jest metodą interpolacji w układach, w których używa się opisu Lagrange'a, tj. takich gdzie układ odniesienia porusza się razem z cieczą. Przy wykorzystaniu SPH pole, które jest zdefiniowane w konkretnych punktach (w położeniach gdzie znajdują się cząsteczki) może zostać opisane dla dowolnego miejsca w przestrzeni. W tym celu wykorzystuje się następujące rówanie:\\

\begin{samepage}
{\sc Równanie interpolacyjne SPH: }
\begin{equation}
A_S(\boldsymbol{r}) = \sum\limits_{j} m_j {A_j \over\rho_j}W(\boldsymbol{r}-\boldsymbol{r}_j, h),
\label{eqn:sph}
\end{equation}
gdzie:

\begin{itemize}[noitemsep,topsep=0pt,parsep=0pt,partopsep=0pt]
\renewcommand\labelitemi{--}
\item $ A_S $ - interpolacja pola w pozycji $\boldsymbol{r}$,
\item $j$ - cząsteczki po których następuje iteracja,
\item $m_j$ - masa $j$-tej cząsteczki,
\item $\boldsymbol{r}_j$ - pozycja $j$-tej cząsteczki,
\item $\rho_j$ - gęstość $j$-tej cząsteczki,
\item $A_j$ - wartość pola w punkcie $\boldsymbol{r}_j$,
\item $W(\boldsymbol{r}-\boldsymbol{r}_j, h)$ - jądro wygładzające dla j-tej cząsteczki (tj. jej odległości od pozycji $\boldsymbol{r}$: $\| \boldsymbol{r}-\boldsymbol{r}_j \|$).
\end{itemize}
\end{samepage}
\vspace{3ex}
Przy wykorzystaniu powyższego równania da się opisać poszczególne składniki równania Navier'a-Stokes'a \eqref{eqn:navier_stokes_lagrangian} opisane w rozdzale \eqref{subsec:navier_stokes_ss}. Opis ten znajduje się w następnym rozdziale.
\par

\paragraph{}
Istotną właściwością SPH jest łatwość wyliczania pochodnych przestrzennych w postaci gradientu i laplasjanu pola skalarnego. Gradient pola $A_S(\boldsymbol{r})$ w notacji SPH zapisuje się jako:
\begin{equation}
\nabla A_S(\boldsymbol{r}) = \sum\limits_{j} m_j {A_j \over\rho_j} \nabla W(\boldsymbol{r}-\boldsymbol{r}_j, h),
\label{eqn:sph_grad}
\end{equation}
lub w precyzyjniejszej formie gradient pola skalarnego można zapisać (za \cite{kelager06}) jako:
\begin{equation}
\nabla A_S(\boldsymbol{r}) = \rho \sum\limits_{j} m_j \left( {A \over \rho^2} + {A_j \over \rho_j^2} \right) \nabla W(\boldsymbol{r}-\boldsymbol{r}_j, h).
\label{eqn:sph_grad_prec}
\end{equation}
Laplasjan pola $A_S(\boldsymbol{r})$ zapisuje się jako:
\begin{equation}
\nabla^2 A_S(\boldsymbol{r}) = \sum\limits_{j} m_j {A_j \over\rho_j} \nabla^2 W(\boldsymbol{r}-\boldsymbol{r}_j, h).
\label{eqn:sph_lap}
\end{equation}
\par

\subsubsection{Jądra wygładzające}

\paragraph{}
Funkcja $W(\boldsymbol{r}, h)$ to jądro wygładzające (ang. smoothing kernel) o promieniu $h$. Taka funkcja musi spełniać dwa następujące warunki:
\begin{itemize}
\item normalizacja: $$\int_{\Omega} W(\boldsymbol{r}, h) d\boldsymbol{r} = 1;$$
\item zbieżność do delty Diraca przy $h$ dążącym do 0: $$\lim_{h\to0} W(\boldsymbol{r}, h) = \delta(\boldsymbol{r}).$$
\end{itemize}
\vspace{3ex}
\noindent
Oprócz powyższych warunków funkcja jądra wygładzającego musi być dodatnia: $$W(\boldsymbol{r}, h) \ge 0$$ aby mogła pełnić rolę uśredniającą. Kiedy dodamy jeszcze jeden warunek funkcji jądra - symetrię: $$W(\boldsymbol{r}, h) = W(\boldsymbol{-r}, h)$$ wtedy interpolacja jest drugiego stopnia precyzji, to znaczy że błąd przybliżenia sumowania w równaniu~\eqref{eqn:sph} jest równy $O(h^{2})$ lub lepszy. Jądro zanika dla promienia większego od $\boldsymbol{r}$: $$W(\boldsymbol{r}, h) = 0, \| \mathbf{r} \| > h.$$
\par

\paragraph{}
Dla zastosowań SPH, w oparciu o powyższe założenia, projektuje się różne funkcje jąder wygładzających. Od nich w dużym stopniu zależy stabilność, dokładność i wydajność metody. Za idealną funkcje jądra uważa się funkcją Gaussa. Jej obliczenie jest jednak czasochłonne (przez zawartą funkcję eksponent) dlatego w praktyce stosuje się prostsze funkcje przypominające tą krzywą.\\
\indent Do większości obliczeń korzysta się z jądra wygładzającego opisanego stosunkowo tanią obliczeniowo funkcją wielomianową $W_{poly6}$:

\begin{equation}
W_{poly6}(\boldsymbol{r}, h) = {315 \over 64 \pi h^9}
\begin{cases}
\left(h^2 - \| \mathbf{r} \|^2\right)^3 & 0 \le \| \mathbf{r} \| \le h \\
0 & \| \mathbf{r} \| > h
\end{cases}
\end{equation}
\begin{equation}
\nabla W_{poly6}(\boldsymbol{r}, h) = {945 \over 32 \pi h^9}
\begin{cases}
\mathbf{r} \left(h^2 - \| \mathbf{r} \|^2\right)^2 & 0 \le \| \mathbf{r} \| \le h \\
0 & \| \mathbf{r} \| > h
\end{cases}
\end{equation}
\begin{equation}
\nabla^2 W_{poly6}(\boldsymbol{r}, h) = {945 \over 32 \pi h^9}
\begin{cases}
\left(h^2 - \| \mathbf{r} \|^2\right)\left(\| \mathbf{r} \|^2 - {3 \over 4}(h^2 \| \mathbf{r} \|^2)\right) & 0 \le \| \mathbf{r} \| \le h \\
0 & \| \mathbf{r} \| > h
\end{cases}
\end{equation}

Powyższe jądro wykorzystywane jest przy interpolacji gęstości, ciśnienia oraz pola kolorowego (dla obliczeń powierzchni). Przy obliczaniu gradientu ciśnienia oraz siły lepkości, z racji natury oddziaływania tych sił, konieczne jest zastosowanie innych jąder wygładzających.\\
\indent W przypadku zastosowania jądra $W_{poly6}$ przy obliczaniu siły ciśnienia cząsteczki mają tendencję do zlepiania się. Dla małych odległości pomiędzy cząsteczkami gradient tego jądra wykorzystywany do obliczania ciśnienia pomiędzy cząsteczkami dąży do zera i siła odpychająca cząsteczki zanika. Ten problem rozwiązuje się korzystając z innego jądra wygładzającego - $W_{spiky}(\boldsymbol{r}, h)$:
\begin{equation}
W_{spiky}(\boldsymbol{r}, h) = {15 \over \pi h^6}
\begin{cases}
\left(h - \| \mathbf{r} \|\right)^3 & 0 \le \| \mathbf{r} \| \le h \\
0 & \| \mathbf{r} \| > h
\end{cases}
\end{equation}
\begin{equation}
\nabla W_{spiky}(\boldsymbol{r}, h) = -{45 \over \pi h^6}
\begin{cases}
{\mathbf{r} \over {\| \mathbf{r} \|}} \left(h - \| \mathbf{r} \|\right)^2 & 0 \le \| \mathbf{r} \| \le h \\
0 & \| \mathbf{r} \| > h
\end{cases}
\end{equation}
Gradient tego jądra pozostaje niezerowy dla argumentów w pobliżu zera, co generuje odpowiednie siły odpychające. Spełnia również pozostałe warunki konieczne dla jąder wygładzających.\\
\indent Do obliczenia siły lepkości konieczne jest wprowadzenie jeszcze jednego jądra wygładzającego. Lepkość jest zjawiskiem, które - spowodowane tarciem - prowadzi do rozproszenia energii kinetycznej płynu w postaci ciepła. Laplasjan takiego jądra musi być dodatni aby zjawisko lepkości było poprawnie odwzorowywane. Stosuje się tutaj jądro wygładzające $W_{viscosity}(\boldsymbol{r}, h)$ opisane następującym wzorem:
\begin{equation}
W_{viscosity}(\boldsymbol{r}, h) = {15 \over 2 \pi h^3}
\begin{cases}
\left( -{ {\| \mathbf{r} \|}^3 \over 2 h^3 } + { {\| \mathbf{r} \|}^2 \over h^2 } + { h \over 2 {\| \mathbf{r} \|} } - 1 \right) & 0 \le \| \mathbf{r} \| \le h \\
0 & \| \mathbf{r} \| > h
\end{cases}
\end{equation}
\begin{equation}
\nabla W_{viscosity}(\boldsymbol{r}, h) = {15 \over 2 \pi h^3}
\begin{cases}
\mathbf{r} \left( -{ 3{\| \mathbf{r} \|} \over 2 h^3 } + { 2 \over h^2 } - { h \over 2 {\| \mathbf{r} \|}^3 } \right) & 0 \le \| \mathbf{r} \| \le h \\
0 & \| \mathbf{r} \| > h
\end{cases}
\end{equation}
\begin{equation}
\nabla^2 W_{viscosity}(\boldsymbol{r}, h) = {45 \over \pi h^6}
\begin{cases}
\left( h - {\| \mathbf{r} \|} \right) & 0 \le \| \mathbf{r} \| \le h \\
0 & \| \mathbf{r} \| > h
\end{cases}
\end{equation}
\par

\paragraph{}

\par
