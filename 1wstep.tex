
%\begin{figure}[h!]
%\centering
%\includegraphics[width=\textwidth]{chaos}
%\caption{test}
%\label{fig:test1}
%\end{figure}

%\begin{lstlisting}
%\lstinputlisting[language=C++, firstline=9, lastline=18, caption=Painter.cpp;9:18, label=Painter.cpp]{Painter.cpp}
%\lstlistingname{Painter.cpp;9:18}
%\end{lstlisting}

\section{Wstęp}

\subsection{Motywacja}
\paragraph{}
Symulacja płynów jest stosunkowo złożonym i skomplikowanym problemem. W ostatnim dziesięcioleciu staje się jednak coraz bardziej popularnym tematem opracowań naukowych. Obliczeniowa mechanika płynów (ang. \textbf{C}omputational \textbf{F}luid \textbf{D}ynamics) wykorzystuje metody numeryczne oraz stale rosnącą moc obliczeniową komputerów do rozwijania tej dyscypliny. Powodem złożoności zachowania płynów jest skomplikowany mechanizm działania różnych zjawisk składających się na mechanikę płynów (zasadę zachowania energii, pędu i momentu pędu dla ośrodka ciągłego) takich jak lepkość, ciśnienie, napięcie powierzchniowe.
\par
Symulacje płynu można przeprowadzać w dwojaki sposób: offline i online. Metody offline zapewniają największą dokładność obliczeń. Oprogramowania stosujące te techniki wykorzystywane są m.in. w przemyśle motoryzacyjnym, lotniczym i kosmicznym gdzie precyzja kalkulacji jest bardzo istotna. Metody dokładne sprowadzają się do rozwiązania równań Naviera-Stokesa.
\par
W ramach tej pracy wykonano symulacje zachowania płynów w oparciu o metodę online -- SPH (Smoothed Particle Hydrodynamics), która wykorzystuje podejście znane z dynamiki molekularnej i używa wariantu dynamiki Newtonowskiej z parametrami, które mają symulować zachowanie cieczy realnej. To podejście sprawdza się w symulacjach w czasie rzeczywistym lub sytuacjach gdzie wymagana jest interakcja z płynem. Metody online wykorzystywane są również we wspomnianych powyżej gałęziach przemysłu, jednak są przydatne jedynie na etapie projektowania. Poza tym metody online znajdują zastosowanie w symulacjach medycznych lub grach wideo.
\par

\subsection{Mój wkład -- założenia}
\paragraph{}
W niniejszej pracy moim celem było zaprezentowanie metody zdolnej do symulacji płynu w czasie rzeczywistym. Głównym założeniem pracy była możliwość interakcji z płynem w stosunkowo dużej skali przy zachowaniu satysfakcjonującej wydajności. Mój wybór, przy sugestii promotora, padł na metodę, która zdobywa w ostatnim dziesięcioleciu najwięcej rozgłosu: Smoothed Particle Hydrodynamics (pol. wygładzonej hydrodynamiki cząstek).
\par
Osobnym krokiem w ramach symulacji płynu jest jego wizualizacja, na którą składają się m.in. elementy: wyznaczenia powierzchni płynu oraz jej renderowania. Ten fragment często pochłania najwięcej czasu symulacji i kluczowym jest znalezienie efektywnego rozwiązania. W niniejszej pracy został przedstawiony jeden z wielu opracowanych sposobów - jednak zdaniem autora również jeden z najciekawszych - na rozwiązanie tego problemu.
\par
